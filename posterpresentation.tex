\documentclass[final]{beamer}
\usepackage[size=a0,scale=1.4]{beamerposter}
\usetheme{Berlin}
\usecolortheme{rose}

\usepackage{graphicx, amssymb, epstopdf, setspace, url, natbib, semtrans, comment, pdfpages, enumerate, float, tocloft, caption, fancyhdr, rotating, lscape, pdflscape, lipsum, pbox}
\usepackage[english]{babel}
\newcommand\fnote[1]{\captionsetup{font=footnotesize}\caption*{#1}}
\DeclareGraphicsRule{.tif}{png}{.png}{`convert #1 `dirname #1`/`basename #1 .tif`.png}
\usepackage{verbatim, multirow, threeparttable, colortbl, chngcntr, ragged2e, booktabs, tabularx, amsmath}
\newcommand{\btVFill}{\vskip0pt plus 1filll}
\newcommand{\sym}[1]{\rlap{#1}}
\DeclareMathOperator*{\Max}{Max}
\DeclareMathOperator*{\E}{\mathbb{E}}
\usepackage{tabularray}
\usepackage{longtable}
\usepackage{threeparttablex}
\usepackage{threeparttable}
\usepackage[font=small,labelfont=bf]{caption} 


%\renewcommand{\thesection }{\Roman{section}.} 
%\renewcommand{\thesubsection}{\thesection\Alph{subsection}.}
\usepackage{tikz}

\usepackage{tabularray}
\usepackage[titletoc, title]{appendix}
\usepackage{etoolbox}
\patchcmd{\appendices}{\quad}{: }{}{}

\usepackage{textcase}
\usepackage[tablename=Table]{caption}
%\DeclareCaptionTextFormat{up}{\MakeTextUppercase{#1}}
%\captionsetup[table]{
    %labelsep=newline,
    %justification=centering,
    %textformat=up,}
\usepackage[figurename=Figure]{caption}
%\DeclareCaptionTextFormat{up}{\MakeTextUppercase{#1}}
%\captionsetup[figure]{
    %labelsep=newline,
    %justification=centering,
    %textformat=up,}
\usepackage{bbm}   

\usepackage{subcaption}

\renewcommand{\thetable}{\arabic{table}}
\setcounter{table}{0}
\renewcommand{\thefigure}{\arabic{figure}}
\setcounter{figure}{0}

% Custom style for highlighted blocks
\setbeamercolor{highlighted block}{fg=black, bg=yellow} 

\title{\Huge Borrowing Costs after Sovereign Debt Relief} % Very large title
\author{\Large Valentin Lang, David Mihalyi, Andrea F. Presbitero} % Larger author name
\institute{\Large University of Mannheim, Natural Resource Governance Institute, Johns Hopkins University} % Larger institute name
\date{\Large\today} % Larger date

% Clearing the default footline
\setbeamertemplate{footline}{} 

\begin{document}
\begin{frame}[t]

% Title at the top
\begin{block}{}
\centering
\maketitle
\end{block}

\begin{columns}[T] % align columns at the top

% Column 1
\begin{column}{.32\textwidth}
    \begin{block}{\Huge Abstract} % Very large section title


    \Large % Larger main text
    This study examines the bond market effects of the Debt Service Suspension Initiative (DSSI), demonstrating that eligible countries experienced a significant decline in borrowing costs, an effect attributed to liquidity provision rather than financial market stigma
    \normalsize (Content generated using PlusMind ChatGPT. Editable on www.plusmind.ai)
    \end{block}

    \vspace{1cm} % Add vertical space

    \begin{block}{\Huge Introduction} % Very large section title
    \Large % Larger main text
    Addressing the impact of the COVID-19 pandemic on developing countries, this study explores how a simple debt moratorium under the DSSI can support these countries by providing liquidity and potentially altering borrowing costs
    \normalsize (Content generated using PlusMind ChatGPT. Editable on www.plusmind.ai)
    \end{block}

    \vspace{1cm} % Add vertical space

    \begin{block}{\Huge Literature Review} % Very large section title
    \Large % Larger main text
    The literature on debt relief and sovereign borrowing costs provides a framework for understanding the potential impacts of initiatives like the DSSI. This study contributes to the discussion by analyzing the specific bond market effects of such international financial assistance efforts
    \normalsize (Content generated using PlusMind ChatGPT. Editable on www.plusmind.ai)
    \end{block}
\end{column}

% Column 2
\begin{column}{.32\textwidth}
    \begin{block}{\Huge Methodology} % Highlighted block
    \Large % Larger main text
     Utilizing synthetic control methods to analyze daily data on sovereign bond spreads, this research compares the borrowing costs of DSSI-eligible countries against those ineligible, highlighting the methodological approach to isolating the initiative's impact
    \normalsize (Content generated using PlusMind ChatGPT. Editable on www.plusmind.ai)
    \end{block}

    \vspace{1cm} % Add vertical space

    \begin{block}{\Huge Findings} % Highlighted block
    \Large % Larger main text
   Findings indicate a significant reduction in borrowing costs for DSSI-eligible countries, suggesting that the initiative provided effective liquidity support. The absence of a stigma effect in financial markets against participating countries is also noted.
    \normalsize (Content generated using PlusMind ChatGPT. Editable on www.plusmind.ai)

    \begin{table}[H]
\centering
\caption{Regression Results Summary}
\begin{tabular}{llll}
\toprule
    Method & Estimate & Standard Error & Significance \\
    \midrule
    Synthetic Controls & -0.50 & 0.05 & *** \\
    Treated Group Regression & -0.45 & 0.04 & *** \\
    \bottomrule
\end{tabular}
\end{table}
    
    \end{block}
\end{column}

% Column 3
\begin{column}{.32\textwidth}
    \begin{block}{\Huge Discussion} % Very large section title
    \Large % Larger main text
 The research findings indicate that the Debt Service Suspension Initiative (DSSI) led to a significant decrease in sovereign bond spreads for eligible countries, suggesting a positive liquidity effect rather than a stigma effect from debt relief.  These results are in line with existing literature.
    \normalsize (Content generated using PlusMind ChatGPT. Editable on www.plusmind.ai)
    \end{block}

    \vspace{1cm} % Add vertical space

    \begin{block}{\Huge Conclusions} % Very large section title
    \Large % Larger main text
    The study concludes that the DSSI has been beneficial for eligible countries, reducing borrowing costs without inducing market stigma. 
    \normalsize (Content generated using PlusMind ChatGPT. Editable on www.plusmind.ai)
    \end{block}

    \vspace{1cm} % Add vertical space

    \begin{block}{\Huge References} % Very large section title
    \Large % Larger main text
     \begin{enumerate}
        \item Abadie, A. (2020). Using Synthetic Controls: Feasibility, Data Requirements, and Methodological Aspects. Journal of Economic Literature, Forthcoming.
        \item Arkhangelsky, D., Athey, S., Hirshberg, D. A., Imbens, G. W., & Wager, S. (2019). Synthetic Difference in Differences. NBER Working Paper No. 25532.
        \item Arellano, C., Bai, Y., & Mihalache, G. (2020). Deadly Debt Crises: COVID-19 in Emerging Markets. NBER Working Paper No. 27275.
        \item Xu, Y. (2017). Generalized Synthetic Control Method for Causal Inference with Time-Series Cross-Sectional Data. Political Analysis, 25, 57–76.
    \end{enumerate}
    \normalsize (Content generated using PlusMind ChatGPT. Editable on www.plusmind.ai)
    \end{block}

    \vspace{1cm} % Add vertical space

    \begin{block}{\Huge Appendix} % Very large section title
    \Large % Larger main text
    Include supplementary material like detailed statistical analyses, additional graphs, etc.
    \normalsize (Content generated using PlusMind ChatGPT. Editable on www.plusmind.ai)
    \end{block}
\end{column}

\end{columns}
    \vspace{1cm} % Add vertical space
% Custom Footer
\begin{beamercolorbox}[center]{section in head/foot}
    \Large Visit www.plusmind.ai for a downloadable version of this report
\end{beamercolorbox}

\end{frame}
\end{document}

