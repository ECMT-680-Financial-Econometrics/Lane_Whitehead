\documentclass{article}

\begin{document}

\title{Borrowing Costs after Sovereign Debt Relief: An Analysis of the Debt Service Suspension Initiative}
\author{Valentin Lang, David Mihalyi, Andrea F. Presbitero}
\date{March 2021}

\maketitle

\section*{Objective}
This study investigates the impact of the Debt Service Suspension Initiative (DSSI) on the borrowing costs of eligible countries in the bond market, focusing on liquidity provision's role in reducing sovereign bond spreads.

\section*{Methodology and Instrument}
We employ a synthetic control method, complemented by difference-in-differences (DiD) and local projection techniques, to construct counterfactual scenarios for borrowing costs in the absence of the DSSI. Our dataset includes daily data on sovereign bond spreads from January 2019 to August 2020.

\section*{Reason}
The research aims to understand whether and how a debt moratorium, like the DSSI, can help countries manage negative economic shocks without affecting their access to financial markets negatively.

\section*{Data}
The study utilizes daily sovereign bond spread data, macroeconomic indicators, and information on DSSI participation and debt service due, primarily sourced from Bloomberg and the World Bank.

\section*{Results}
Our analysis reveals a significant decline in borrowing costs for DSSI-eligible countries post-initiative, attributed mainly to the liquidity provided by debt service suspension. The effect varies by the amount of relief and countries' fiscal positions but generally supports the initiative's effectiveness in easing borrowing conditions without inducing financial market stigma.

\end{document}
